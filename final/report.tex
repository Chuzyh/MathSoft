\documentclass{article}
\usepackage{ctex, hyperref}
\usepackage{graphicx}
\usepackage{amsmath}
\usepackage{hyperref}
\usepackage{amsfonts,amssymb}
\usepackage[top=2cm, bottom=2cm, left=2cm, right=2cm]{geometry}  
\usepackage{algorithm}  
\usepackage{algpseudocode}
\usepackage{float}  
\usepackage{listings} 
\usepackage{cite}
\usepackage{tikz}
\usetikzlibrary{arrows,shapes,chains}
\tikzstyle{decision} = [diamond, draw, fill=orange!80, text width=5.0em, text centered, node distance=3.1cm, ] 
\tikzstyle{document} = [rectangle, draw, fill=blue!30, text width=4.7em, text centered, node distance=2.5cm, ] 
\tikzstyle{block1} = [rectangle, draw, fill=green!50, %here, we have chosen another block for the different color
text width=4.5em, text centered, rounded corners, node distance= 3cm, minimum height=3em] % you can create as many different shapes to make your diagram more creative and attractive, depending on the requirements

\tikzstyle{block} = [rectangle, draw, fill=yellow!50, text width=4.5em, text centered, rounded corners, node distance=2.25cm, minimum height=4em]
\tikzstyle{line} = [draw, -latex']
\tikzstyle{cloud} = [draw, ellipse, text width= 2.9em, fill=red!50, node distance=2cm, minimum height=3em]

\tikzstyle{ioi} = [trapezium, draw, trapezium right angle=120, rounded corners, fill=blue!60, node distance=2.8cm, minimum height=2.7em]
 \tikzstyle{io} = [trapezium, draw, trapezium right angle=110, rounded corners, fill=red!20, node distance=1.9cm, minimum height=2.9em]   % the draw command here is used to draw the boundary of mentioned shape.
 \tikzstyle{arrow} = [thick,->,>=stealth]

\bibliographystyle{plain}
\newtheorem{theorem}{定理}
\title{最终项目作业: 一维非线性方程的求根}


\author{褚朱钇恒 \\ 信息与计算科学 3200104144}

\begin{document}

\maketitle
\begin{abstract}
     一维非线性方程$f(x)=0$不一定有精确的代数解,但可以求出数值解。本文主要介绍了二分法和牛顿法两种求解一维非线性方程方法,并比较两者在求解不同方程的收敛性和收敛速度。
     
\end{abstract}
\section{引言}
     求解一维非线性方程
     \begin{equation}
          f(x)=0;
          \label{eq::1}
     \end{equation}
     即确定$x^*\in[a,b]$使$f(x^*)=0$,其中$f$是定义在[a,b]上的连续非线性函数,满足$f(x^*)=0$的$x^*$成为方程(\ref{eq::1})的根,也称为$f(x)$的零点。
\section{数学理论}
     \begin{theorem}{介值定理}
          \label{the::intermediate}
          如果$f(x)$在区间$[a,b]$上连续且$f(a)f(b)<0$,则至少存在一点$\eta\in[a,b]$使$f(\eta)=0$

     \end{theorem}
     \begin{theorem}{泰勒公式}
          \label{the::taylor}
          把$f(x)$在$x_0$点附近展开成Taylor级数,得到
          $$f(x)=f(x_0)+(x-x_0)f^{'}(x_0)+(x-x_0)^2f^{''}(x_0)+\cdots$$
     \end{theorem}
\section{问题分析}
     \subsection{二分法}
     由定理\ref{the::intermediate}可以得出二分法求解一维非线性方程的想法,即每次将区间中点视为近似根,如果误差超过了要求,则考虑将原左右端点中与中点的$f$值异号的点作为新的端点,然后在子区间上继续迭代。

     \subsection{Newton法}
     由定理\ref{the::taylor},取其线性部分则有$f(x_0)-f^{'}{x_0}(x-x_0)=0$,其解为$x_1=x_0-\frac{f(x_0)}{f^{'}(x_0)}$,故得到迭代序列$x_{n+1}=x_n-\frac{f(x_n)}{f^{'}(x_n)}$
     
\section{算法}
     根据上一节的分析,我们不难得出二分法和Newton法的具体流程。
\begin{figure}
\centering
\begin{tikzpicture}[node distance = 1.8cm, auto] 
     \node [cloud] (init) {输入$f,a,b$};
     \node [block, below of = init](D){$mid=\frac{a+b}{2}$};
     \node [decision, below of = D](F){判断$f(a)f(mid)$是否$<0$};
     \node [block, below of = F,yshift=-1cm](G){$b=mid$};
     \node [block, left of = G](H){$a=mid$};
     
     \node [decision, below of = G](C){判别是否停止迭代};
     \node [cloud, below of = C,yshift = -1cm](I){输出结果};
     \node (tmp) [coordinate, right of =C,xshift=0.5cm]{};
     \node (tmp2) [coordinate, right of =D,xshift=0.5cm]{};
     \node (tmp3) [coordinate, left of =F,xshift=-0.5cm]{};
                     
     \path [line] (init) -- (D); 
     \path [line] (D) -- (F); 
     \draw [arrow] (F) -- node[anchor=east] {yes} (G);
     \path [line] (G) -- (C); 
     \draw [arrow] (C) -- node[anchor=east] {no} (I);
     \path [line] (H) -- (C); 
     
     \draw [arrow] (C.east) --(tmp)-- node[anchor=east] {yes}(tmp2) -- (D.east);
     \draw [arrow] (F.west) --(tmp3)-- node[anchor=east] {no} (H); 
  
     
 \end{tikzpicture}
 \begin{tikzpicture}[node distance = 1.8cm, auto] % the command node distance is important as it determines the space or the length of the arrow between different blocks.
     \node [cloud] (init) {输入$f,f^{'},x_0$};
     \node [block, below of = init](J){$n=1$};
     \node [block, below of = J](D){$x_{n}=x_{n-1}-\frac{f(x_{n-1})}{f^{'}(x_{n-1})}$};

     \node [block, below of = D,yshift=-1cm](G){$n=n+1$};
     
     \node [decision, below of = G](C){判别是否停止迭代};
     \node [cloud, below of = C,yshift = -1cm](I){输出结果};
     \node (tmp) [coordinate, right of =C,xshift=0.5cm]{};
     \node (tmp2) [coordinate, right of =D,xshift=0.5cm]{};
     \node (tmp3) [coordinate, left of =F,xshift=-0.5cm]{};
                     
     \path [line] (init) -- (J); 
     \path [line] (J) -- (D); 
     \path [line] (D) -- (G); 
     \path [line] (G) -- (C); 
     \draw [arrow] (C) -- node[anchor=east] {no} (I);
     
     \draw [arrow] (C.east) --(tmp)-- node[anchor=east] {yes}(tmp2) -- (D.east);

     
 \end{tikzpicture}
 \caption{二分法和Newton法算法流程图}
 \label{algo::1}
\end{figure}
\bibliography{quote}
\end{document}
