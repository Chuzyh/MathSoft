\documentclass{article}
\usepackage{ctex, hyperref}
\usepackage{graphicx}
\usepackage{amsmath}
\usepackage{hyperref}
\usepackage{amsfonts,amssymb}
\usepackage[top=2cm, bottom=2cm, left=2cm, right=2cm]{geometry}  
\usepackage{algorithm}  
\usepackage{algpseudocode}
\usepackage{float}  
\usepackage{listings} 
\usepackage{cite}
\usepackage{tikz}
\usetikzlibrary{arrows,shapes,chains}
\tikzstyle{decision} = [diamond, draw, fill=orange!80, text width=5.0em, text centered, node distance=3.1cm, ] 
\tikzstyle{document} = [rectangle, draw, fill=blue!30, text width=4.7em, text centered, node distance=2.5cm, ] 
\tikzstyle{block1} = [rectangle, draw, fill=green!50, %here, we have chosen another block for the different color
text width=4.5em, text centered, rounded corners, node distance= 3cm, minimum height=3em] % you can create as many different shapes to make your diagram more creative and attractive, depending on the requirements

\tikzstyle{block} = [rectangle, draw, fill=yellow!50, text width=4.5em, text centered, rounded corners, node distance=2.25cm, minimum height=4em]
\tikzstyle{line} = [draw, -latex']
\tikzstyle{cloud} = [draw, ellipse, text width= 2.9em, fill=red!50, node distance=2cm, minimum height=3em]

\tikzstyle{ioi} = [trapezium, draw, trapezium right angle=120, rounded corners, fill=blue!60, node distance=2.8cm, minimum height=2.7em]
 \tikzstyle{io} = [trapezium, draw, trapezium right angle=110, rounded corners, fill=red!20, node distance=1.9cm, minimum height=2.9em]   % the draw command here is used to draw the boundary of mentioned shape.

\bibliographystyle{plain}
\newtheorem{theorem}{定理}
\title{作业六: 求解一元二次方程}


\author{褚朱钇恒 \\ 信息与计算科学 3200104144}

\begin{document}

\maketitle
\section{问题描述}
形如
$$
ax^2+bx+c=0,a,b,c \in \mathbb{R}
$$
的方程称为一元二次方程,求$x_0 \in \mathbb{R}$满足该方程等价于求$f(x)=ax^2+bx+c$的零点。
\section{问题分析}
     首先将二次项系数化为1,得到$$x^2+\frac{b}{a}x+\frac{c}{a}=0$$
     
     然后使用配方法将原方程变形得到$$x^2+2\frac{b}{2a}x+(\frac{b}{2a})^2=-(\frac{b}{2a})^2-\frac{c}{a}$$

     $$(x+\frac{b}{2a})^2=-\frac{b^2}{4a^2}-\frac{c}{a}$$

     最后得到$$x=\frac{-b\pm\sqrt{b^2-4ac}}{2a}$$

     由于$b^2-4ac$的正负性影响了方程实根的数量,故称其为判别式$\Delta$,通过以下示意图不难看出$\Delta$与$f(x)$零点数量即方程实根数量之间的关系。

     \begin{figure}[H]
          \begin{center}
               \input{eg1.tex}
          \end{center}
          \caption{$\Delta$的正负性和零点数量之间的关系}
     \end{figure}
\section{求解方法}
     通过上述分析,我们不难得出求解求解一元二次方程的算法:即先根据判别式$\Delta$的不同判断实根的数量,再代入求根公式。下面的流程图就是该算法的具体表述。
\begin{figure}
\centering
\begin{tikzpicture}[node distance = 1.8cm, auto] % the command node distance is important as it determines the space or the length of the arrow between different blocks.
     % the command given below are the place of nodes
     \node [cloud] (init) {输入$a,b,c$};
     \node [decision, below of = init](C){判别式$\Delta$的值};
     \node [block, below of = C,yshift=-1cm](D){$\Delta>0$,方程有两个不相等的实根};
     \node [document, below of = D](F){求解方程的根$x=\frac{-b\pm\sqrt{\Delta}}{2a}$};
     \node [cloud, below of = F](G){输出结果};
     \node [block, right of = D](H){$\Delta<0$,方程无实根};
     \node [block, left of = D](J){$\Delta=0$,方程有两个相等的实根};
     \node (tmp) [coordinate, above of =J,yshift=-0.3cm]{};
     \node (tmp2) [coordinate, above of =H,yshift=-0.3cm]{};
                     
     \path [line] (init) -- (C); 
     \path [line] (C) -- (D); 
     \path [line] (D) -- (F); 
     \path [line] (F) -- (G); 
     \path [line] (H) |- (G.east); % |- is used for making the rectangular arrow
     \path [line] (C.south) --(tmp) -- (J.north); % to position the arrow, you can use north, south, east, and west directions
     \path [line] (C.south) --(tmp2) -- (H.north); % to position the arrow, you can use north, south, east, and west directions
  
     \path [line] (J) |- (F.west); 
     
 \end{tikzpicture}
 \caption{算法流程图}
 \label{algo}
\end{figure}
\bibliography{quote}
\end{document}
