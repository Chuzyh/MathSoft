\documentclass{article}
\usepackage{ctex, hyperref}
\usepackage{graphicx}
\usepackage{amsmath}
\usepackage{hyperref}
\usepackage{amsfonts,amssymb}
\usepackage[top=2cm, bottom=2cm, left=2cm, right=2cm]{geometry}  
\usepackage{algorithm}  
\usepackage{algpseudocode}
\usepackage{float}  
\usepackage{listings} 
\title{作业二: 配置Linux工作环境}


\author{褚朱钇恒 \\ 信息与计算科学 3200104144}

\begin{document}

\maketitle

\section{选择Linux 发行版}
    Ubuntu 20.04.4 LTS
\section{安装的软件}
    \begin{itemize}
        \item python
        \item python3
        \item grep
        \item texlive
        \item synaptic
        \item make
        \item cmake
        \item Vscode
        \item libboost
        \item trilinos
        \item dx
        \item git
        \item ssh
        \item clash
    \end{itemize}
\section{简单的配置}


\begin{verbatim}
#!/bin/bash
g++ std.cpp -o std
g++ NeedCheck.cpp -o wa
for input_file in $(ls *.in); do
	./std <$input_file >out1
	./wa <$input_file >out2
	if diff out1 out2 -b -B; then
		echo "Accept" ${input_file%.*}
	else 
		echo "WrongAnswer" ${input_file%.*}
	fi
done
\end{verbatim}

    接下来运行脚本就可以看到错误代码通过了哪些测试点,在哪些测试点输出了错误的答案。

\end{document}
