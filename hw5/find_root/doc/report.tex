\documentclass{article}
\usepackage{ctex, hyperref}
\usepackage{graphicx}
\usepackage{amsmath}
\usepackage{hyperref}
\usepackage{amsfonts,amssymb}
\usepackage[top=2cm, bottom=2cm, left=2cm, right=2cm]{geometry}  
\usepackage{algorithm}  
\usepackage{algpseudocode}
\usepackage{float}  
\usepackage{listings} 
\usepackage{cite}
\bibliographystyle{plain}
\newtheorem{theorem}{定理}
\title{作业五: 安装最新版本的 gsl}


\author{褚朱钇恒 \\ 信息与计算科学 3200104144}

\begin{document}

\maketitle
\section{对roots.c的分析}
     对于函数$F(x)=x^2-5$使用brent method迭代法求出它的一个在$[0,5]$中的一个解。

     输出结果
\begin{verbatim}
using brent method
 iter [    lower,     upper]      root        err  err(est)
    1 [1.0000000, 5.0000000] 1.0000000 -1.2360680 4.0000000
    2 [1.0000000, 3.0000000] 3.0000000 +0.7639320 2.0000000
    3 [2.0000000, 3.0000000] 2.0000000 -0.2360680 1.0000000
    4 [2.2000000, 3.0000000] 2.2000000 -0.0360680 0.8000000
    5 [2.2000000, 2.2366300] 2.2366300 +0.0005621 0.0366300
Converged:
    6 [2.2360634, 2.2366300] 2.2360634 -0.0000046 0.0005666
\end{verbatim}
     不难发现$[lower,upper]$表示迭代得出的解的范围,$root$为当前迭代出的解,$err$为当前解和真实解的差,$err(est)$为当前解的范围的大小。
\end{document}
