\documentclass{ctexart}

\usepackage{graphicx}
\usepackage{amsmath}
\usepackage{amsfonts,amssymb}

\title{作业一: 实对称矩必可在实数域上相似对角化的叙述与证明}


\author{褚朱钇恒 \\ 信息与计算科学 3200104144}

\begin{document}

\maketitle


这是一个来自线性代数领域的问题.
\section{问题描述}
问题叙述如下: 任何一个$n$阶的实对称阵均可在$\mathbb{R}$上相似对角化,且存在$n$阶的正交矩阵$U$,使$U^TAU$为对角阵。

\section{证明}

我们只要证明存在$n$阶的正交矩阵$U$,使$U^TAU$为对角阵即可。为此,我们对矩阵的阶做归纳,若$A$为一阶方阵,则它已经相似对角化,令$U=(1)_{1\times 1}$即得证。

设已证明任何一个$n-1$是对称矩阵都存在想要的正交矩阵$U_1$,使得$U_1^TAU_1$为对角阵,则对于任意一个$n$阶实对称阵$A$,由于$n$阶实对称矩阵有$n$个实特征值。设$\lambda_1$为其中一个特征值,$\xi_1$为$A$的属于$\lambda_1$的一个实特征向量且$|\xi_1|=1$. 用Schmidt正交化方法将$\xi_1$扩充成$\mathbb{R}^n$中的一个标准正交基$\xi_1,\xi_2,\cdots,\xi_n$,则$A\xi_1,A\xi_2,\cdots,A\xi_n$均可以经$\xi_1,\xi_2,\cdots,\xi_n$线性表示,不难验证
\begin{equation}
    \begin{split}
    A\begin{pmatrix} \xi_1 & \xi_2 & \cdots & \xi_n \end{pmatrix}
    &=\begin{pmatrix} A\xi_1 & A\xi_2 & \cdots & A\xi_n \end{pmatrix} \\
    &=\begin{pmatrix} \xi_1 & \xi_2 & \cdots & \xi_n \end{pmatrix}\begin{pmatrix} \lambda_1 & \alpha \\ O &A_1 \end{pmatrix}
    \end{split}
    \label{eq::eq1}
\end{equation}

这里$\alpha$为$n-1$维实行向量,$A_1$为$n-1$阶实方阵。令
$$U_0=\begin{pmatrix} \xi_1 & \xi_2 & \cdots & \xi_n \end{pmatrix}$$

则依$\begin{matrix} \xi_1 & \xi_2 & \cdots & \xi_n \end{matrix}$为$\mathbb{R}^n$的一个标准正交基知$U_0$为正交矩阵。故(\ref{eq::eq1})等价于
\begin{equation}
    \begin{split}
    U^T_0AU_0
    &=\begin{pmatrix} \lambda_1 & \alpha \\ O &A_1 \end{pmatrix}
    \end{split}
    \label{eq::eq2}
\end{equation}

由于(\ref{eq::eq2})等式左端为实对称的,故右端也是实对称的,从而$\alpha=\theta$,$A_1$为$n-1$阶的实对称阵。依归纳假设知,存在$n-1$阶的正交矩阵$U_1$与$n-1$阶对角阵$D_1$使$U_1^TA_1U_1=D_1$,令
$$
U=U_0\begin{pmatrix} 1 &   \\   &U_1 \end{pmatrix}
$$

则$U$为正交矩阵,且
\begin{equation}
    \begin{split}
    U^TAU
    &=\begin{pmatrix} 1 &   \\   &U_1^T \end{pmatrix}\begin{pmatrix} \lambda_1 &   \\   &A_1 \end{pmatrix}\begin{pmatrix} 1 &   \\   &U_1 \end{pmatrix}\\
    &=\begin{pmatrix} \lambda_1 &   \\   &D_1 \end{pmatrix}
    \end{split}
\end{equation}

这说明所要证明的结论对于$n$阶实对称阵$A$也成立。依数学归纳法,定理得证。

\end{document}
