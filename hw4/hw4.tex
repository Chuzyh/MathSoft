\documentclass{article}
\usepackage{ctex, hyperref}
\usepackage{graphicx}
\usepackage{amsmath}
\usepackage{hyperref}
\usepackage{amsfonts,amssymb}
\usepackage[top=2cm, bottom=2cm, left=2cm, right=2cm]{geometry}  
\usepackage{algorithm}  
\usepackage{algpseudocode}
\usepackage{float}  
\usepackage{listings} 
\usepackage{cite}
\bibliographystyle{plain}
\newtheorem{theorem}{定理}
\title{作业四: Mandelbrot Set 的生成和探索}


\author{褚朱钇恒 \\ 信息与计算科学 3200104144}

\begin{document}

\maketitle

\begin{abstract}
     Mandelbrot 集是复数的集合 $c$,使$f_{c}(z)=z^{2}+c$迭代时不会发散到无穷大.我们可以用迭代法判断每个像素对应的复数是否属于$c$,然后进行着色,画出$Mandelbrot\ set$的图像。
\end{abstract}
\section{引言}
Mandelbrot 集的图像有了一个精细且无限复杂的边界,随着放大倍数的增加,它会显示出越来越精细的递归细节;在数学上,它的边界是一条分形曲线。我们可以用一张bmp图片对应复平面上的一片区域,这样就让一个像素点对应了一个复数,我们可以根据该复数迭代多少次后模大于2,来选择该像素点的颜色,于是我们便可以画出$Mandelbrot$集的图像。
\section{问题的背景介绍}
Mandelbrot 集起源于复动力学,这是 20 世纪初法国数学家 Pierre Fatou和Gaston Julia首次研究的领域。1978 年,罗伯特·W·布鲁克斯和彼得·马特尔斯基首次定义并绘制了这种分形,作为克莱因群研究的一部分。1980 年 3 月 1 日,Benoit Mandelbrot在纽约约克镇高地的IBM托马斯 J. 沃森研究中心首次看到了该集合的可视化。

1985 年 8 月的《科学美国人》的封面文章向广大读者介绍了计算 Mandelbrot 集的算法。封面由不来梅大学的Peitgen 、Richter 和Saupe创作。Mandelbrot 集在 1980 年代中期作为计算机图形演示变得突出,当时个人计算机变得足够强大,可以以高分辨率绘制和显示该集。\cite{wikipedia_Mandelbrot}

\section{数学理论}
\begin{theorem}{判断是否不收敛}
     从0开始迭代的轨道趋于无穷当且仅当在某个点它的模数$>2$.
\end{theorem}

\section{算法}
\begin{algorithm}  [H]
     \caption{判断$c$对应的$f_c$是否有界}  
     \begin{algorithmic}[1] %每行显示行号  
         \Require   $c,N$ 需要判断的复数和迭代次数
         \Ensure   $iterator\_times,is\_bound$ 迭代到模数大于2的次数,是否在给定的迭代次数内是否存在某个点模大于2.
         
         \State $z \gets 0$
         \State $is\_bound \gets 1$
          
          \For{$i=1;i\le N;i++$}
               \State $z \gets z*z+c$
               \If{$|z|>2$}
                 \State $iterator\_times \gets i$
                 \State $is\_bound \gets 0$
                 \State break
               \EndIf
         \EndFor

     \end{algorithmic}  
 \end{algorithm}  
 \begin{algorithm}  [H]
     \caption{判断每个像素点的颜色信息}  
     \begin{algorithmic}[1] %每行显示行号  
         \Require   $ox,oy,dim$图片中心对应的复数$ox+oy\textbf{i}$,图片横向所表示的范围$[-dim,dim]$ 
         \State $N\gets \frac{255}{3}$
          \For {每个像素点}
               \State $c \gets $该像素点对应的复数
               
               \State use Algorithm 1
               \If{$is\_bound$}
                    \State 该节点的RGB值$\gets 0$
               \Else
                    \State 该节点的RGB值$\gets 255-iterator\_times*3$
               \EndIf
         \EndFor

     \end{algorithmic}  
 \end{algorithm}  
\section{数值算例}
(请先使用\verb|make|命令得到可执行文件)

首先我们来看一下整体图像

使用命令\verb|./test all.bmp 0 0 1|得到图像
\begin{figure}[H]
     \centering
     \includegraphics[scale=0.28]{./image/all.bmp}
     \caption{整体图像}
\end{figure}
然后我们以$(-1.4,0)$为中心放大,用更小的精度渲染更小范围的图像,容易发现其有美丽的分形结构。
使用命令\verb|./test part.bmp -1.4 0 0.05|得到图像
\begin{figure}[H]
     \centering
     \includegraphics[scale=0.28]{./image/part.bmp}
     \caption{局部图像}
\end{figure}

似乎灰度变化的分界线不太明显,我们不妨改成用颜色分界,这样更容易看出图像的分形特征。
使用命令\verb|./test part.bmp -1.4 0 0.05 colorful|得到图像
\begin{figure}[H]
     \centering
     \includegraphics[scale=0.26]{./image/col.bmp}
     \caption{彩色图像}
\end{figure}
\section{结论}
     我们用C++程序画出了Mandelbrot Set的简单示意图,用观察图像的方式感性理解了一下这个集合的特点。

     容易发现在该图像的一个部分内, 能找到大小更小而和整体形状相似的图形结构. 这说明Mandelbrot Set是一个分形结构.
\section{附注}
     本项目用到的\verb|.cpp|文件都放在\verb|src|文件夹中,用到的\verb|.h|文件都放在\verb|include|文件夹中;您可以使用\verb|make|命令得到名为\verb|test|的可执行文件,也可以使用\verb|make report|命令得到该pdf文档。
\bibliography{quote}
\end{document}
